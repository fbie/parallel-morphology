\chapter{Discussion}
\label{chpt:discussion}

\section{Contributions}
\label{sec:discussion-contribution}

As a major result, we can conclude, from the findings made in
section~\ref{sec:experiments-area-opening}, that sub-image area opening with
thread-local sorting, using a fine-grained optimistic locking scheme for the
union-find implementation, is the most efficient and robust parallel algorithm
presented in this thesis.

Model checking, as shown in chapter~\ref{chpt:correctness}, showed that the
parallel algorithm compute the same result as the sequential algorithm by
\citet{Meijster2002Comparison}. It benefits from an inherent heuristic, which is
an expectable large distance between the pixel start indices on the image array
for the independent threads. Therefore, lock contention on single trees is low
in most of the cases. There are two cases, however, which pose worst case
scenarios to the algorithm. These are image gradients in $x$- or $y$-direction,
which dominate the image structure. In these cases, the algorithm is either (1)
forced to work in a sequential fashion, which is slower than the sequential
baseline, due to the threading overhead, or (2) encounters high lock contention,
due to level components extending excessively across sub-image borders.

It is interesting to see that parallel algorithms with the same formal
complexity behave very differently in practice. For example, the shared sorting
algorithms, which run in $O(k + n)$, do not perform well in any case, while the
most naive algorithm in $O(kn)$ (that is, sub-image filtering without sorting)
is second to best in every experiment.

Effects such as false sharing \cite{Bolosky1993False} and lock contention have a
huge influence on the actual performance. So does the input structure, apart
from its size, as we saw in the results obtained by independent union-find
experiments, shown in section~\ref{sec:experiments-uf-results}. Here, the
connectivity of the input graph made a large difference in terms of performance.

A similar algorithm for labeling connected flat components on gray-scale images,
based on union-find was proposed by \citet{Hesselink2001Concurrent}. The
algorithm they proposed does \emph{not} compute area opening, which is the major
difference to this thesis. Their approach uses no shared union-find data
structure, but a message passing system through which ownership of disjoint sets
is moved between threads. This means that Hesselink et al.'s approach does not
benefit from implicitly shared data structures. Still, their results show that
their algorithm scales linearly with the number of hardware threads. Even though
their algorithm and the ones developed in this thesis are closely related, they
are only comparable to some degree.

Sub-image parallel area opening, in its current version, does not scale linearly
with the number of hardware threads. In the presented experimental results, it
first begins to show a noticeable improvement over the sequential algorithm when
running on four threads, and already stops improving at around eight to ten
threads. This upper bound will presumably increase with the size of the
input. Nevertheless, this also means that, in order to gain a substantial
performance improvement, we need to provide the algorithm with a sufficiently
large image to filter. This suggests that the sequential algorithm by
\citet{Meijster2002Comparison} already performs optimally.

The number of algorithms developed in this thesis is comparably large in order
to give a comprehensive overview over efficient parallelization strategies for
area opening. Knowing which algorithms yield good performances, only analyzing a
few of those with less parameters (i.e. parallel union-find variants, different
implementations of auxiliary data structures, etc.) will provide further
interesting insights.

\section{Future Work}
\label{sec:future-work}

Taken the insights found in this thesis, there are quite a number of
possibilities for further research directions and implementation work. For
example, a more remote task would be bug-fixing and extending JPF to be able to
fully model check the presented algorithms.

In order to perform a more general verification of area opening algorithms, it would be
helpful to find the ``sweet spot'' between operational correctness conditions,
as presented in section~\ref{sec:correctness-area-opening-properties-ao}, and
black-box testing conditions, as shown in
section~\ref{sec:correctness-experimental}. This could help to identify and
verify even more efficient parallel area opening algorithms and also to verify
generalizations of parallel area opening. Such properties should also focus more
on consistency. A node's parent value is invariant as soon as it is equal or
greater to zero, so that $I: ~ 0 \leq \inlinecode{parent(find(x))}$ and $\{I\} ~
c ~ \{I\}$, where $c$ is any function defined in
section~\ref{sec:morphology-algorithm}.  This invariant implies the absence of
deletion in the union-find data structure and should be part of an extended
formal validation of parallel area opening to ensure consistency.

As already pointed out, there exists a generalized variant of area opening,
namely morphological attribute opening \cite{Breen1996Attribute}. The sub-image
area opening algorithm could, in principle, easily be extended to attribute
openings, as done by \citet{Wilkinson2000Fast} for the sequential algorithm
\cite{Meijster2002Comparison}. Additionally, \citet{Meijster2001Fast} showed
that union-find based area opening can directly be projected onto the
computation of area granulometry, or area pattern spectra
\cite{Meijster2002Comparison}. Area pattern spectra are histograms of the size
distribution over a given image \cite{MohanaRao2001Areagranulometry}. Computing
these spectra is helpful in order to estimate the average size of elements that
are expected to pose the majority of an image for further segmentation.

It would be interesting to assess the performance of sub-image parallel area
opening more deeply. In order to avoid the issues of micro benchmarks in managed
languages like Java, one could re-implement the algorithm in a medium-level
language, such as C or C{}\verb!++!. Additionally, the algorithm could be
extended to also filter 3D voxel images. Based on the findings that large images
benefit more from the parallel algorithms, the performance on three dimensional
data can be expected to be very fast. Furthermore, the performance of sub-image
parallel area opening should be compared systematically to the performance of
parallel Max-Tree computation \cite{Wilkinson2008Concurrent}.

Also, the number of regional maxima on images could then be controlled to a
greater extend, for example by controlling the number of level components
explicitly, and there would be more room for more exhaustive performance
experiments.

Furthermore, the sub-image filtering algorithm needs to be stabilized against
worst case image scenarios. One approach could be to simply use the sequential
algorithm in such cases, based on a heuristic analysis of the image
structure. Analyzing the image entirely once, and then filtering it, is, on
average, probably just as slow as always simply running the parallel
algorithm. Instead, one could examine how the image can be split up between
threads, as to avoid sub-image patterns, which are prone to contention with
regard to image structure. For example, instead of defining intervals on the
pixel array, in which each thread may access pixels, one could assign a
rectangle of pixels to each thread that is also limited on the $x$ axis.

Lastly, one should mention that modern GPUs provide a fast platform for parallel
image algorithms and it is therefore tempting to think about an area opening
algorithm for GPUs. An approach to implementing sub-image parallel area opening
on GPUs could be a union of the algorithm presented in this thesis and the
cross-thread message passing union-find algorithm for flat component labeling,
as presented by \citet{Hesselink2001Concurrent}.

%%% Local Variables:
%%% mode: latex
%%% TeX-master: "main"
%%% End:
