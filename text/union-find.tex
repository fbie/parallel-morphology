\chapter{Shared Union-Find Data Structures}
\label{chpt:union-find}

This chapter describes two different algorithms for a shared union-find
\cite{Tarjan1983Data} data structure. The first one is the wait-free union-find
algorithm by \citet{Anderson1994Waitfree}. The other one relies
on fine-grained locking per disjoint tree \cite{Berman2010Multicore}.

The following sections contain many different figures, that convey variants of
parallel union-find. In order to explain the algorithms thoroughly, I believe
that it is easiest for the reader to look at code examples. This is especially
true, because the many variants of the union-find algorithm are very similar and
the differences are very subtle, if explained in plain English. I deliberately
omit introducing the original union-find by \citet{Tarjan1983Data} in favor of
focusing on the concurrent algorithms.

\section{A Wait-Free Implementation}
\label{sec:union-find-wf}

As described in section \ref{sec:introduction-concurrency}, wait-free concurrent
algorithms have a number of desirable properties, compared to blocking
algorithms. Therefore, \citet{Anderson1994Waitfree} developed a wait-free
union-find algorithm, based on the \emph{CAS} primitive. Their implementation
uses ranking heuristics to balance trees, in order to maintain short chains
between leafs and roots.

The wait-free union-find algorithm represents nodes in an array of records,
called \inlinecode{record} and again accessible through a function of the same
name. Each \inlinecode{Record} contains an integer representation of the rank
of a tree and an (atomic) integer pointing to the index of the record's parent
in the array. The implementation of \inlinecode{Record} is shown in
figure~\ref{fig:union-find-wf-union-record}.

\begin{figure}
  \centering
  \lstinputlisting[linerange={record0-record1}]{../parallel-union-find/src/dk/itu/parallel/unionfind/WaitFree.java}
  \caption[The \inlinecode{Record} class, as used in wait-free union-find.]{The
    \inlinecode{Record} class, as used in wait-free union-find. Maintaining the
    rank and the pointer to the node's parent in this object allows an atomic
    update of both variables using \emph{CAS}.}
  \label{fig:union-find-wf-union-record}
\end{figure}

The parent pointer initially points to the index of the record itself, to
indicate that it is a root. It is here implemented as an atomic reference, to
make implementing a wait-free path-compressing \inlinecode{find} easy. If
\inlinecode{find} was not to compress paths, it could simply traverse up to the
root. This would be wait-free already, as \inlinecode{find} does not guarantee
anything but to return a node that was root at some point in time
\cite{Anderson1994Waitfree}. To compress paths in a wait-free manner,
\citet{Anderson1994Waitfree} implement their find using path-halving:

\lstinputlisting[style=inline, linerange={find0-find1}]{../parallel-union-find/src/dk/itu/parallel/unionfind/WaitFree.java}

The parent pointer of each visited node is set to the node's grandparent, if no
other thread modified the parent pointer in the meantime. If another thread
interfered, the failed \emph{CAS} is simply ignored and traversal continues.

Uniting two disjoint trees is a matter of updating the parent pointer of a node,
as well as its rank. To perform this update atomically,
\citet{Anderson1994Waitfree} use the \inlinecode{Record} object associated with
this node. The record of the node, which will be updated, is atomically replaced
by a record pointing to the new record and maintaining a new rank. This update
is performed by the auxiliary function \inlinecode{updateRoot}. It terminates
early, if another thread modified the soon-to-be child node at
\inlinecode{x}. Otherwise, it simply returns to the caller, whether the update
has been successful:

\lstinputlisting[style=inline, linerange={updateroot0-updateroot1}]{../parallel-union-find/src/dk/itu/parallel/unionfind/WaitFree.java}

While this would already be enough to implement a wait-free \inlinecode{union},
\citet{Anderson1994Waitfree} acknowledge, that this can result in long chains,
if multiple threads unite trees, without the rank of the root being incremented
successfully. To counter this, they add a call to \inlinecode{setRoot}, which
performs a \inlinecode{find}-like traversal and compression. Finally,
\inlinecode{setRoot} tries to update the rank of the root by one atomically,
using \inlinecode{updateRoot}:

\lstinputlisting[style=inline, linerange={setroot0-setroot1}]{../parallel-union-find/src/dk/itu/parallel/unionfind/WaitFree.java}

The \inlinecode{union} function, in its entire form, is depicted in figure
\ref{fig:union-find-wf-union}.

\begin{figure}
  \lstinputlisting[linerange={union0-union1,union2-union3,union4-union5}]{../parallel-union-find/src/dk/itu/parallel/unionfind/WaitFree.java}
  \caption[Wait-free implementation of \inlinecode{union}.]{Wait-free
    implementation of \inlinecode{union}. \citet{Anderson1994Waitfree}
    acknowledge, that this implementation of concurrent \inlinecode{union} can
    result in long chains in the disjoint tree structure.}
  \label{fig:union-find-wf-union}
\end{figure}

\subsection{Adaption to Area Opening}
\label{sec:union-find-wf-adaption}

A major flaw of \citeauthor{Anderson1994Waitfree}'s wait-free union-find
algorithm is the spuriously occurring failure of updating ranks, as well as the
temporary inconsistencies between node updates \cite{Anderson1994Waitfree,
  Berman2010Multicore}. Updating the size of a root node, together with updating
references to it, is key to label connected components correctly. Otherwise,
another thread could grow a component, while another is in-between updating the
root pointer and updating the new root's size, which are independent from each
other. This can result in uniting pixel sets based on wrong priorities. Using
\inlinecode{Record} as a container to update rank and parent pointer atomically,
works only for updating both values on the same node and does therefore not
provide a solution to the problem.

This is a manifestation of the missing composition capabilities of
synchronization primitives, as already mentioned in section
\ref{sec:introduction-cas}. \citet{Herlihy2008Art} give a proof, that there is
no wait-free implementation to assign values to multiple, independent fields
(i.e. an $(m,n)$-assignment) consistently.

This means that the wait-free union-find algorithm cannot be directly applied to
our problem.

\subsection{STM Variant}
\label{sec:union-find-wf-stm}

To update two independent nodes atomically when computing area opening, we can
resort to an \emph{STM} version of \citeauthor{Anderson1994Waitfree}'s
algorithm. The \inlinecode{find} routine stays entirely wait-free and still
performs path compression by path-halving, as shown in
figure~\ref{fig:union-find-wf-stm-find}. It is now mandatory to avoid compressing the path
to a negative parent pointer, as that would result in removing a sub-tree from
its root.

\begin{figure}
  \centering
  \lstinputlisting[linerange={find0-find1}]{../parallel-morphology/src/dk/itu/parallel/morphology/unionfind/Transactional.java}
  \caption[Wait-free \inlinecode{find} for area opening.]{Wait-free
    \inlinecode{find} for area opening. \emph{multiverse}'s
    \inlinecode{atomicWeakGet} corresponds to \inlinecode{get} on Java's atomic
    types; \inlinecode{atomicCompareAndSet} corresponds to
    \inlinecode{compareAndSet}. As soon as the \inlinecode{parent} value of a
    node is equal or below zero, it will not be set to below zero again.}
  \label{fig:union-find-wf-stm-find}
\end{figure}

The concept of records is not required any longer. Also, ranking and the
according rank updates are removed, in order to model the correct relation of
pixels to each other \cite{Meijster2002Comparison}. This implies also the
removal of \inlinecode{updateRoot} and \inlinecode{setRoot}.

The wait-free \inlinecode{union}'s while-pattern is still required by the
algorithm, if we use negative values to model component size, while compressing
the paths using a wait-free \inlinecode{find}. Otherwise, we can run into
complications when checking, whether the negated parent value (i.e. the size) is
less than \inlinecode{lambda} (see figure \ref{fig:union-find-wf-stm-union}).

\begin{figure}
  \centering
  \lstinputlisting[linerange={union0-union1,union2-union3,union4-union5}]{../parallel-morphology/src/dk/itu/parallel/morphology/unionfind/Transactional.java}
  \caption[STM \inlinecode{union} for area opening.]{STM \inlinecode{union} for
    area opening. \inlinecode{TxnVoidCallable} is a \emph{multiverse} wrapper
    class, similar to the \inlinecode{Callable} interface. Note that the
    while-loop is still required, in order to catch modifications by wait-free
    \inlinecode{find}.}
  \label{fig:union-find-wf-stm-union}
\end{figure}

Moreover, we can implement conditional deactivation of not-united root nodes. If
a to-be-deactivated root already exceeds \inlinecode{lambda}, it does not need
to be deactivated explicitly. Short-circuiting this operation saves overhead due
to unnecessary modifications by the transaction, which would otherwise come at
the considerable cost of copying and modifying the variable and then committing
the changes \cite{Herlihy2008Art}.

To compare the performance of using an explicit while-loop , I implemented a
general version of \citeauthor{Anderson1994Waitfree}'s union-find with
\emph{STM} modifications, as well as a simple \emph{STM} implementation. The
latter simply wraps the sequential \inlinecode{union} in a single transaction.

\section{An Optimistic Implementation}
\label{sec:union-find-optimistic}

\begin{figure}

  \lstinputlisting[linerange={union0-union1,union2-union3,union4-union5}]{../parallel-union-find/src/dk/itu/parallel/unionfind/Optimistic.java}

  \caption[Optimistic locking implementation of \inlinecode{union}.]{Optimistic
    locking implementation of \inlinecode{union}. The algorithm performs
    \inlinecode{find}, until it has acquired the locks for the actual roots.}
  \label{fig:union-find-optimistic-union}
\end{figure}

Typically, locks are used globally for data structures. This blocks threads from
performing changes, while another thread is modifying some part of the
data. While easier to implement, this leads to high contention, as the progress
of all threads \emph{not} holding the global lock is blocked, even if their
modifications would not collide with those of the active thread. This is not
acceptable for efficient parallel algorithms.

Another approach is optimistic, or fine-grained, locking of resources. A thread
only acquires locks for those parts of a data structure, that it wants to modify
consistently. Writing to an array at different indices is probably the best
example for a data structure, that invites such a locking scheme, as each index
of the array can get assigned a unique lock.

In his thesis, \citet{Berman2010Multicore} developed a union-find algorithm
based on optimistic locking. He claims that this approach outperforms the
wait-free union-find implementation by \citet{Anderson1994Waitfree}. In his
implementation, the disjoint sets are represented by three arrays of the same
length: one array \inlinecode{next}, that, for each node, holds a pointer to
either the parent index or to itself, if it is root. Another array
\inlinecode{rank}, that maintains the rank of each node, and finally an array
\inlinecode{lock}, that maintains a lock for each node in the disjoint set
structure \cite{Berman2010Multicore}.

To update the relation of two trees consistently, \citet{Berman2010Multicore}
uses a (not further described) function \inlinecode{lock\_roots} that locks two
trees in the forest of disjoint trees by acquiring the locks only for the
respective root nodes. Java does not provide a two-object locking
mechanism. Therefore, the Java implementation uses a simple double-lock
algorithm on the \inlinecode{lock} array, which is of type
\inlinecode{ReentrantLock[]}. This algorithm is based on a simple lock algorithm
given by \citet{Herlihy2008Art}. Unlocking both trees, in case acquiring both
locks failed, ensures that no deadlocking occurs, in case a thread only
successfully acquired one lock:

\lstinputlisting[style=inline, linerange={lock0-lock1}]{../parallel-union-find/src/dk/itu/parallel/unionfind/Optimistic.java}

\citet{Berman2010Multicore} proposes a scheme that, once it acquired the locks
for two nodes, checks that these nodes are still roots. The fine-grained locking
makes it possible for the nodes to change their state, while a thread waits to
acquire their locks. In the Java implementation, this locking scheme is
integrated into \inlinecode{union}, as shown in figure
\ref{fig:union-find-optimistic-union}.

Another notable difference to the wait-free algorithm is the path
compression. The optimistic algorithm compresses paths only after successfully
uniting two sets, while still holding the locks \cite{Berman2010Multicore}. This
makes \inlinecode{find} fully wait-free, as no locks must be acquired during
traversal. Path compression is implemented by re-assigning the parent pointer of
each node along the path to the root's index.

\subsection{Adaption to Area Opening}
\label{sec:union-find-optimistic-adaption}

For the optimistic union-find implementation, the changes for using it in area
opening are nearly negligible. Here as well, both pixel's roots must be found
prior to proceeding. Updating the size of a tree at the root node and the parent
pointer of another node, in order to make it to point to the new root, is not an
issue in a lock-based implementation. Again, ranking heuristics are removed
\cite{Meijster2002Comparison}.

\begin{figure}
  \centering
  \lstinputlisting[linerange={compress0-compress1}]{../parallel-morphology/src/dk/itu/parallel/morphology/unionfind/Optimistic.java}
  \caption[The \inlinecode{compress} function for optimistic area opening.]{The
    \inlinecode{compress} function for optimistic area opening. It is only
    executed when the thread holds the corresponding lock for the node at
    \inlinecode{root}.}
  \label{fig:union-find-optimistic-adaption-compress}

  \lstinputlisting[linerange={union0-union1,union2-union3}]{../parallel-morphology/src/dk/itu/parallel/morphology/unionfind/Optimistic.java}
  \caption[Optimistic \inlinecode{union} for area opening.]{Optimistic
    \inlinecode{union} for area opening. The paths are compressed w.r.t. success
    or failure of uniting the two sets.}
  \label{fig:union-find-optimistic-adaption-union}
\end{figure}

In order to determine whether found nodes are still roots, it is sufficient to
check, whether their \inlinecode{parent} value is below zero. As soon as a
node's \inlinecode{parent} value is equal to or greater than zero, it will not
be re-set to below zero, since this union-find structure does not support
deletion. The same goes for the wait-free implementation of \inlinecode{find}.

Another modification is the compression of paths (see
figure~\ref{fig:union-find-optimistic-adaption-compress}), which depends on the
success or failure of conditional union. If the trees have been united, both
trees are compressed to the same root. Otherwise, their respective original
roots are used as compression targets. The complete algorithm for optimistic
\inlinecode{union}, adapted to area opening, is shown in
figure~\ref{fig:union-find-optimistic-adaption-union}.

%%% Local Variables:
%%% mode: latex
%%% TeX-master: "main"
%%% End:
