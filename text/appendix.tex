\chapter{Auxiliary Algorithms}
\label{chpt:auxiliary-algorithms}

\section{Bounded Array Queue}
\label{sec:bounded-array-queue}

The implementation of the bounded array queue, which I use for maintaining work
items like pixels or image lines, is lock-free. It maintains two instances of
\inlinecode{AtomicInteger}, that point to the \inlinecode{head} and the
\inlinecode{tail} of the queue (i.e. indices on the pixel array). The array is
always of size $2^n$. This makes it possible to use the bit-wise \emph{and}
operator to compute modulo of the \inlinecode{head} and \inlinecode{tail}
pointers, in order to wrap-around the pointers to point to the beginning of the
underlying array again. The queue is empty, if \inlinecode{head} =
\inlinecode{tail}. When an element is enqueued, \inlinecode{tail} is advanced,
pointing to the next empty array index. When fetching an element from the queue,
\inlinecode{head} is advanced again. Each of these operations has a complexity
of constant time. The implementation of these functions is shown in detail in
figure~\ref{fig:bounded-array-queue}. For brevity, the constructor and other,
rather uninteresting, details are omitted.

\begin{figure}
  \centering
  \lstinputlisting[linerange={queue0-queue1,queue2-queue3,queue4-queue5,queue6-queue7,queue8-queue9}]{../parallel-morphology/src/dk/itu/parallel/morphology/util/queues/ConcurrentArrayQueue.java}
  \caption{The implementation of the bounded array queue.}
  \label{fig:bounded-array-queue}
\end{figure}

\section{Shared Queues}
\label{sec:area-opening-queues} % kept like this to not break references

\begin{figure}
  \centering

  \lstinputlisting[linerange={queues0-queues1,queues2-queues3,queues4-queues5}]{../parallel-morphology/src/dk/itu/parallel/morphology/util/queues/Queues.java}

  \caption{The \inlinecode{Queues} container class.}
  \label{fig:ao-wf-queues}
\end{figure}

The \inlinecode{Queues} data structure maintains two concurrent queues,
\inlinecode{upper} and \inlinecode{lower}, and a level pointer which indicates
the currently filtered gray-scale level. On initialization, all items are
enqueued to \inlinecode{upper}. Calling \inlinecode{pop}, the upper queue is
successively emptied by consuming threads. After the threads are done with their
work on the retrieved item, they put it into the lower queue through
\inlinecode{enqueue}.

Atomically swapping and gray-level decrease is implemented in the auxiliary
function \inlinecode{swapAndDecrement}. Each thread acquires a local reference
of the thread global \inlinecode{Queues} instance. It can then issue a call to
update the global queue, using its local reference to check if it is up to date:

\lstinputlisting[style=inline, linerange={swap0-swap1}]{../parallel-morphology/src/dk/itu/parallel/morphology/util/queues/Queues.java}

Obviously, the swap will fail if the calling thread does not have the most
recent information on the global queues stored locally, so that we can make sure
that it will be swapped only once per level.

\section{Pixel Line Representation}
\label{sec:pixel-line}

\begin{figure}
  \centering
  \lstinputlisting[linerange={line0-line1}]{../parallel-morphology/src/dk/itu/parallel/morphology/filter/QueueLines.java}

  \caption{The \inlinecode{Line} container class implementation.}
  \label{fig:line-class}
\end{figure}

In order to represent a single pixel line, we rely on a container class
\inlinecode{Line}, that wraps an array, which in turn contains, by gray-scale
value sorted, pixel indices. For convenience, the container class also
internally maintains a pointer to the pixel, which will be returned on the next
call to \inlinecode{current}. Furthermore, the function \inlinecode{advance}
moves the pointer and returns true, if the internal pointer is not yet outside
of the array bounds. The same check can be performed by calling
\inlinecode{workLeft}. The entire implementation is depicted in
figure~\ref{fig:line-class}.

\chapter{Raw Data Plots}
\label{chpt:omitted-data}

\begin{figure}[h]
  \centering
  \fullrawrow{\synthone}{ms}{raw-hypothesis-1a}
  \caption{Execution time in miliseconds for \emph{grad-vert}.}
  \fullrawrow{\synthone}{retries}{raw-hypothesis-1a}
  \caption{Number of retries for \emph{grad-vert}.}
\end{figure}

\begin{figure}
  \centering
  \fullrawrow{\synthone}{cache-misses}{raw-hypothesis-1a}
  \caption{Number of cache-misses for \emph{grad-vert}.}
  \fullrawrow{\synthone}{instructions}{raw-hypothesis-1a}
  \caption{Number of issued CPU instructions for \emph{grad-vert}.}
  \fullrawrow{\synthtwo}{ms}{raw-hypothesis-1b}
  \caption{Execution time in milliseconds for \emph{grad-hrz}.}
\end{figure}

\begin{figure}
  \centering
  \fullrawrow{\synthtwo}{retries}{raw-hypothesis-1b}
  \caption{Number of retries for \emph{grad-hrz}.}
  \fullrawrow{\synthtwo}{cache-misses}{raw-hypothesis-1b}
  \caption{Number of cache-misses for \emph{grad-hrz}.}
  \fullrawrow{\synthtwo}{instructions}{raw-hypothesis-1b}
  \caption{Number of issued CPU instructions for \emph{grad-hrz}.}
\end{figure}

\begin{figure}
  \centering
  \fullrawrow{\synththreea}{ms}{raw-hypothesis-2a}
  \caption{Execution time in miliseconds for \emph{noise-fine}.}
  \fullrawrow{\synththreea}{retries}{raw-hypothesis-2a}
  \caption{Number of retries for \emph{noise-fine}.}
  \fullrawrow{\synththreea}{cache-misses}{raw-hypothesis-2a}
  \caption{Number of cache-misses for \emph{noise-fine}.}
\end{figure}

\begin{figure}
  \centering
  \fullrawrow{\synththreea}{instructions}{raw-hypothesis-2a}
  \caption{Number of issued CPU instructions for \emph{noise-fine}.}
  \fullrawrow{\synthfour}{ms}{raw-hypothesis-2b}
  \caption{Execution time in miliseconds for \emph{noise-coarse}.}
  \fullrawrow{\synthfour}{retries}{raw-hypothesis-2b}
  \caption{Number of retries for \emph{noise-coarse}.}
\end{figure}

\begin{figure}
  \centering
  \fullrawrow{\synthfour}{cache-misses}{raw-hypothesis-2b}
  \caption{Number of cache misses for \emph{noise-coarse}.}
  \fullrawrow{\synthfour}{instructions}{raw-hypothesis-2b}
  \caption{Number of issued CPU instructions for \emph{noise-coarse}.}
  \fullrawrow{\synthfive}{ms}{raw-hypothesis-3}
  \caption{Execution time in miliseconds for \emph{flat}.}
\end{figure}

\begin{figure}
  \centering
  \fullrawrow{\synthfive}{retries}{raw-hypothesis-3}
  \caption{Number of retries for \emph{flat}.}
  \fullrawrow{\synthfive}{cache-misses}{raw-hypothesis-3}
  \caption{Number of cache misses for \emph{flat}.}
  \fullrawrow{\synthfive}{instructions}{raw-hypothesis-3}
  \caption{Number of issued CPU instructions for \emph{flat}.}
\end{figure}

\begin{figure}
  \centering
  \fullrawrow{\natone}{ms}{raw-nat-1}
  \caption{Execution time in miliseconds for \emph{rbc}.}
  \fullrawrow{\natone}{retries}{raw-nat-1}
  \caption{Number of retries for \emph{rbc}.}
  \fullrawrow{\natone}{cache-misses}{raw-nat-1}
  \caption{Number of cache misses for \emph{rbc}.}
\end{figure}

\begin{figure}
  \centering
  \fullrawrow{\natone}{instructions}{raw-nat-1}
  \caption{Number of issued CPU instructions for \emph{rbc}.}
  \fullrawrownat{\nattwo}{ms}{raw-nat-2}
  \caption{Execution time in miliseconds for \emph{facade}.}
  \fullrawrownat{\nattwo}{retries}{raw-nat-2}
  \caption{Number of retries for \emph{facade}.}
\end{figure}

\begin{figure}
  \centering
  \fullrawrownat{\nattwo}{cache-misses}{raw-nat-2}
  \caption{Number of cache misses for \emph{facade}.}
  \fullrawrownat{\nattwo}{instructions}{raw-nat-2}
  \caption{Number of issued CPU instructions for \emph{facade}.}
  \fullrawrownat{\natthree}{ms}{raw-nat-3}
  \caption{Execution time in miliseconds for \emph{mammo}.}
\end{figure}

\begin{figure}
  \centering
  \fullrawrownat{\natthree}{retries}{raw-nat-3}
  \caption{Number of retries for \emph{mammo}.}
  \fullrawrownat{\natthree}{cache-misses}{raw-nat-3}
  \caption{Number of cache misses for \emph{mammo}.}
  \fullrawrownat{\natthree}{instructions}{raw-nat-3}
  \caption{Number of issued CPU instructions for \emph{mammo}.}
\end{figure}

%%% Local Variables:
%%% mode: latex
%%% TeX-master: "main"
%%% End:
