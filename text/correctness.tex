\chapter{Correctness}
\label{chpt:correctness}

This chapter is devoted to the correctness of the algorithms presented in
chapter~\ref{chpt:area-opening}. An inherent problem of parallel and concurrent
programs is the non-determinism of interleavings between independent
threads. Because of this, standard unit-testing of parallel code, as done with
sequential code, is not enough to ensure the absence of errors.

Instead, there exist a number of formal methods to verify parallel programs. One
of the first formal frameworks to reason about correctness in concurrent
programs developed, was \emph{Communicating Sequential Programs} (CSP)
\cite{Hoare1978Communicating}. CSP allows manual reasoning and proving of
programs, but there also exist digital implementations \cite{Roscoe1997Theory}.

Another, more practical approach is model checking \cite{Visser2003Model}. The
advantage of model checking is that it can easily be fully automated. However,
model checking suffers from analyzing huge state-spaces. In this chapter, I will
focus on showing the absence of errors in the parallel area opening algorithms
through model checking by explicit state enumeration using \emph{Java
  Pathfinder} \cite{Visser2003Model}. Additionally, I will briefly touch upon
the possibilities of unit-testing parallel area opening.

\section{Experimental Correctness Verification}
\label{sec:correctness-experimental}

In contemporary software engineering, unit-testing is an often used tool to
assure that independent parts of a program behave correctly. Input values must
be provided manually, which means that testing is basically experimental
correctness verification and by no means exhaustive -- this is true for both
sequential and concurrent programs. Nevertheless, it is a good starting point in
practice. Especially parallel algorithms can, as a first step, be tested on a
single thread to assure conceptual correctness.

Due to the already pointed out non-determinism of thread interleavings -- caused
either by true hardware parallelism or by the intransparent heuristics of a
thread scheduler -- testing of truly parallel algorithms is practically
impossible. Errors might seem to occur randomly, or so spuriously that they are
not found during testing \cite{Goetz2006Java}. To increase the chance to hit
erroneous execution paths, tests can of course be repeated automatically with a
huge variety of input data. Still, this is not exhaustive. Thread scheduling is
non-deterministic, but not entirely random. Executing a concurrent program
multiple times does not guarantee to execute different thread
interleavings. Therefore, more elaborate methods are required, in order to
verify the correctness of parallel programs.

For development purposes, I implemented an experimental black-box testing suite,
which asserts a single post-condition for the parallel algorithms. Recall from
section~\ref{sec:morphology-area-opening} that $\gamma^a_{\lambda}$ denotes area
opening and $f$ a projection from a subset of $\mathbb{R}^2$ into
$\mathbb{R}^*$. Let, moreover, $\rho^a_{\lambda}$ denote any parallel variant of
area opening. We assert that:

\begin{equation}
  \label{eq:correctness-experimental-post}
  \gamma^a_{\lambda}(f) = \rho^a_{\lambda}(f)
\end{equation}

Again, due to the non-determinism of thread scheduling, this assertion might
only fail occasionally for incorrect programs.

\section{Java Pathfinder}
\label{sec:correctness-jpf}

Java Pathfinder (JPF) is an explicit state enumerator and model checker
\cite{Visser2003Model}. JPF has been used to verify concurrent real-life
\emph{NASA} programs and has discovered serious bugs \cite{Visser2004Test}. The
model checker uses search heuristics to execute all relevant states of a
program. It reports erroneous operations, such as accessing fields on
\inlinecode{null} objects or division by zero. Additionally, programs can be
annotated with pre- and post-conditions, as well as class invariants, to model
more complex correctness properties.

JPF can be seen as a drop-in JVM replacement and has been especially designed to
uncover concurrency related bugs \cite{Visser2003Model}. Any Java program can be
run directly in JPF, with the exception of programs that use native code. JPF
aware programs (those that are annotated with properties) benefit most from
model checking and, theoretically, no code-changes are required when moving the
model checked target code to a production system running a typical JVM.

\subsection{Symbolic Execution}
\label{sec:correctness-jpf-symbolic}

Symbolic execution is a method of model checking that computes values of
variables, in order to execute each path of the program once
\cite{Khurshid2003Generalized}. Thereby, it is not necessary to execute the
system under test (SUT) with many different input variables to trigger each path
once, possibly re-taking already executed and checked program paths. Symbolic
variables are assigned as needed, to trigger all possible program paths. This is
trivial for data types like Boolean, as there are only two possible values to
assign. In the case of numeric data types, JPF uses linear programming solvers
to compute appropriate values, depending on the program's control flow.

The state space of the area opening algorithm is dependent on the image
structure. If we focus on the possible number of orderings of the pixels of an
image containing $n$ pixels, times the valid range of values for $\lambda \in
[0, n]$, the input space becomes $O(nn!)$. Symbolic execution can effectively
reduce the checking time, by triggering the execution of each combination of
\inlinecode{union} and \inlinecode{uniteNeighbors} once instead of using real
pixel values

\subsection{Current State of Implementation}
\label{sec:correctness-jpf-impl}

JPF, in its most recent version (\emph{jpf-core}, commit 1155:1f82eee5f139),
provides automatic state exploration and performs the most basic correctness
checks. Further modules allow to annotate Java code with pre- and
post-conditions and class invariants (\emph{jpf-aprop}, commit 64:0f5c04466e1c)
and model checking with symbolic values (\emph{jpf-symbc}, commit
583:b1bdfb401e9f)\footnote{All JPF modules have been checked out on the
  27.04.2014.}.

Annotations are limited to accessing class fields and method
parameters. Functions cannot be called from within annotations, since functions
might have side effects. JPF has no notion of \emph{pure} functions. (There is a
\inlinecode{@SandBox} annotation, but it has no effect in this regard.) In order
to model check parallel area opening, we require access to a number of auxiliary
functions, to find the root of a connected component, to get a pixel's
gray-value and so on (see
section~\ref{sec:correctness-area-opening-properties}). One method to implement
more complex properties through a \inlinecode{Property} interface is to use the
\inlinecode{satisfies} directive. The JPF implementation to check properties in
this way seems to be buggy. Another major bug in the JPF implementation is an
error, that occurs when using the Java \inlinecode{Logger} class. This makes it
currently impossible to verify any STM variants of the algorithms using
\emph{multiverse}. Even though fixing bugs in JPF is out of scope of this
project and the verification of STM algorithms is therefore identified as future
work, I performed some minor changes, in order to correct JPF's error reporting.

JPF has a vast number of configurable options. There is an active community
around JPF and it is subject to active research \cite{Visser2003Model,
  Khurshid2003Generalized, Visser2004Test}. The configuration process is very
trial-and-error driven. JPF is by no means a light-weight tool for ``drive-by''
software verification.

\section{Verifying Area Opening}
\label{sec:correctness-area-opening}

\subsection{Correctness Properties}
\label{sec:correctness-area-opening-properties}

In order to verify that all algorithms behave the same, it is necessary to
define correctness properties, which are universal to the family of union-find
based area opening algorithms. The following properties are universal and all
algorithms presented in chapter~\ref{chpt:area-opening} must respect these. We
need to verify that (1) the pixels are filtered in the correct order and (2)
that the connected components of two pixels are always united with respect to
the correct conditions.

Because we do not want to alter the program state when executing assertions, we
need to introduce a pure version of \inlinecode{find} that does not perform path
compression. The pure implementation of \inlinecode{find} is recursive and does
not have any side-effects:

\lstinputlisting[style=inline, linerange={findRec0-findRec1}]{../parallel-morphology/verify/dk/itu/parallel/morphology/verify/unionfind/UnionFindVerifyHarness.java}

All invocations of \inlinecode{find} in the remainder of this chapter refer to
the pure, recursive implementation.

\subsubsection{Area Opening}
\label{sec:correctness-area-opening-properties-ao}

As already mentioned in chapter~\ref{chpt:area-opening}, in order to guarantee
implicit scanning of the regional maxima, the algorithms need to filter the
pixels in \emph{decreasing gray-scale order} \cite{Meijster2002Comparison}. This
is an operational correctness property. In the sequential algorithm, this can be
ensured by simply sorting pixels by gray-scale value and then iterating the
pixel array from bright to dark pixels.

Obviously, this is harder to achieve in parallel programs, since we need to
ensure that all threads synchronize on the decrease of the current gray-level
value. Otherwise, it might occasionally occur that one thread proceeds to scan a
pixel of lower gray-value and unites it with a connected component, that thereby
reaches a size of $\lambda$, whereas the component should have instead been
united with a different pixel of higher gray-level first and thereby would
already have reached $\lambda$. This pattern can occur, because of
non-deterministic delays of threads between retrieving a pixel and calling
\inlinecode{uniteNeighbors} on it and results in wrongly grown regions.

In order to assure that pixels are scanned in correct order across independent
threads, we can formulate a simple invariant for \inlinecode{uniteNeighbors}:
its collected input over program execution must be a function of time $t$ to
gray-value, $g: \mathbb{R}^* \rightarrow \mathbb{R}^*$, which is
\emph{monotonically decreasing}, so that $g(t) \leq g(t - 1)$.

To use another, more imperative model, let $Q$ be the set of all already
filtered pixels and $P$ of the pixels yet to process. Initially, $Q = \emptyset$
and $P = I$, where $I$ is the set of all pixels in the image. Let moreover $f: I
\rightarrow \mathbb{R}^*$ denote the gray-value of a pixels (recall this
definition from chapter~\ref{sec:morphology-area-opening}). After each
invocation of \inlinecode{uniteNeighbors}, $Q \leftarrow Q \cup \{c\}$ and $P
\leftarrow P \setminus \{c\}$ are assigned atomically:

\begin{equation}
  \label{eq:correctness-area-opening-properties-ao-inv}
  J: \forall q \in Q, \forall p \in P, f(q) \geq f(p)
\end{equation}

For an actual implementation of this check, it is more convenient to only
maintain a pointer to the last encountered gray-level, $\theta$, and assign
$\theta \leftarrow f(c)$ after each invocation of
\inlinecode{uniteNeighbors}. The invariant in
equation~\ref{eq:correctness-area-opening-properties-ao-inv}, as well as the
property of monotonic decrease, are Markov chains: the state is only dependent
on the last encountered pixel, as it is the \emph{infimum} of all pixels in $Q$
w.r.t gray-level. Therefore, referencing $\theta$ in the invariant is
sufficient:

\begin{eqnarray}
  \label{eq:correctness-area-opening-properties-ao-inv2}
  J': & \theta \geq f(c) \\
   & \{J'\} ~ \inlinecode{uniteNeighbors(c, lambda)} ~\{J'\}
\end{eqnarray}

\subsubsection{Asymmetric Union-Find}
\label{sec:correctness-area-opening-properties-uf}

Until now, we have not seen any special properties of the asymmetric area
opening union-find algorithm. Instead, we will use the sequential algorithm and
the fact that already united components may not be separated again in order to
deduce pre- and according post-conditions for \inlinecode{union}.

The rules for uniting two connected components are simple and can directly be
inferred from the code (see again
figure~\ref{fig:morphology-algorithm-union}). We encounter essentially three
cases: (1) the pixels \inlinecode{n} and \inlinecode{c} are already in the same
set, (2) \inlinecode{find(n)} and \inlinecode{find(c)} are of the same
gray-level, or the size of the connected component of \inlinecode{n} is less
than $\lambda$, or (3) \inlinecode{find(n)} and \inlinecode{find(c)} are of
different gray-level and the size of the connected component of \inlinecode{n}
exceeds $\lambda$.

For each case, consistency is mandatory. The post-conditions of
\inlinecode{union} are dependent on the initial state. In the first case, we must assure that both pixels remain
in the same set after calling \inlinecode{union} on them:

\begin{eqnarray}
  P_1: & \inlinecode{find(n)} = \inlinecode{find(c)} \\
  Q_1: & \inlinecode{find(n)} = \inlinecode{find(c)} \\
  & \{P_1\} ~ \inlinecode{union(n, c)} ~ \{Q_1\}
  \label{eq:correctness-area-opening-properties-uf-1}
\end{eqnarray}

In the case that both pixels fulfill the requirements for uniting them, we need
to make sure that no other thread tampered with the connected components of
\inlinecode{n} and \inlinecode{c} in an inconsistent way. We can do this, by
verifying that the sum of the sizes of both components is the same, before and
after executing \inlinecode{union}. Of course, the roots of both pixels are
required to be the same after uniting them. Let \inlinecode{old} denote the
old-operator, i.e. return the state of a variable before it was modified. Let
now furthermore, $C_x \subseteq I$ denote the set of pixels in the current
connected component of a pixel $x \in I$. This results in the following
post-condition:

\begin{eqnarray}
  P_2: & \neg P_1 \wedge ( f(\inlinecode{find(n)}) = f(\inlinecode{find(c)}) ~ \vee ~ |C_n| < \lambda) \\
  Q_2: & \inlinecode{find(n)} = \inlinecode{find(c)} \wedge |C_n| = \inlinecode{old}(|C_n|) + \inlinecode{old}(|C_c|) \\
  & \{P_2\} ~ \inlinecode{union(n, c)} ~ \{Q_2\}
  \label{eq:correctness-area-opening-properties-uf-2}
\end{eqnarray}

The third case is the direct consequence of $P_2$ not being fulfilled. The
pixels have not been united beforehand, they are not at level and $C_n$ exceeds
$\lambda$. Therefore, $C_c$ must be deactivated and the connected components of
both pixels must remain disjoint.

\begin{eqnarray}
  P_3: & \neg P_1 \wedge \neg P_2 \\
  Q_3: & \inlinecode{find(n)} \neq \inlinecode{find(c)} \wedge |C_c| \geq \lambda \\
  & \{P_3\} ~ \inlinecode{union(n, c)} ~ \{Q_3\}
  \label{eq:correctness-area-opening-properties-uf-3}
\end{eqnarray}

Finally, the following defines the complete correctness property for asymmetric
\inlinecode{union}:

\begin{eqnarray}
  \label{eq:correctness-area-opening-properties-uf}
  P: & P_1 \vee P_2 \vee P_3 \\
  Q: & Q_1 \vee Q_2 \vee Q_3 \\
  & \{P\} ~ \inlinecode{union(n, c)} ~ \{Q\}
\end{eqnarray}

\subsection{Verification}
\label{sec:correctness-area-opening-verification}

Using JPF to verify a complex algorithm, such as parallel union-find based area
opening, poses a number of obstacles, as already outlined in
section~\ref{sec:correctness-jpf-impl}. Because of the bugs in and limitations
of the JPF implementation, the pre- and post-conditions defined in
section~\ref{sec:correctness-area-opening-properties} can not be expressed as
JPF annotations. As a work-around, I implemented sub-classes of the algorithms
under test that call the methods of their super classes and internally and
atomically perform Java assertions before and after. JPF can be configured to
handle failed assertions as property violations. In order not to modify the
state of the program during observation (for example, calling \inlinecode{find}
during checking correctness properties can compress the path of a disjoint tree,
see the beginning of section~\ref{sec:correctness-area-opening-properties}), I
use pure auxiliary implementations of methods that perform
optimization. Additionally, I prune the state space explored, using the
\inlinecode{Verify} API \cite{Visser2003Model}. This API can also be used to
force JPF to execute a sequence of commands atomically -- thread scheduling is
simply disabled in the JVM when JPF executes atomic sequences.

Even though the model checking is performed using symbolic execution, the memory
consumption and the time required for model checking \emph{all} possible program
states is enormous. This means that, in order to achieve some usable results,
the time used for model checking must be limited.

\begin{table}
  \centering
  \makebox[\textwidth][c]{
    \begin{tabular}{ll}
      \hline
      \hline
      \textbf{Algorithm} & \textbf{Acronym} \\
      \hline
      Shared sorting (bucket sort) & \emph{block-bucket} \\
      Shared sorting (counting sort) & \emph{block-counting} \\
      \hline
      Sort-free pixel queues (\citet{Michael1996Simple} queue) & \emph{pixel-queues-msq} \\
      Sort-free pixel queues (bounded array queue) & \emph{pixel-queues-array} \\
      Sort-free sub-image filtering & \emph{split} \\
      \hline
      Non-parallel line queues (\citet{Michael1996Simple} queue) & \emph{line-queues-msq} \\
      Non-parallel line queues (bounded array queue) & \emph{line-queues-array} \\
      Non-parallel counting sort sub-image filtering & \emph{split-counting} \\
      \hline
      Simplified, sequential algorithm by \citet{Meijster2002Comparison} & \emph{sequential} \\
      Sequential algorithm with bogus \inlinecode{union} & \emph{bogus-union} \\
      Sequential algorithm with bogus main loop & \emph{bogus-loop} \\
      \hline
      \hline
    \end{tabular}
  } %makebox
  \caption{Algorithms and their corresponding acronyms.}
  \label{tab:algorithms-acronyms}
\end{table}

This is reasonable, because of an interesting observation. In order to assure
that JPF correctly finds bugs in the algorithms, I implemented a bogus
\inlinecode{union} algorithm that always and unconditionally unites pixels if
they are not already in the same set. Additionally, I added a bogus (linear)
main-loop that provides the input for \inlinecode{uniteNeighbors}. It simply
reverses the order of the sorted pixels and starts at the lowest gray-level. JPF
always quickly finds the violation of the correctness properties after about two
minutes (see
table~\ref{tab:correctness-area-opening-verification-results}). This suggests
that errors are found quickly, but that exploring the entire state space of a
correct program is very time consuming.

\begin{table}[h]
  \centering
  \makebox[\textwidth][c]{
    \begin{tabular}{llrrl}
      \hline
      \hline
      \textbf{Algorithm} & \textbf{Time} & \textbf{Depth} & \textbf{States} & \textbf{Result} \\
      \hline
      \emph{block-bucket} & 01:00:00 & 1079 & 23245 & No errors detected \\
      \emph{block-counting} & 01:00:00 & 1091 & 20426 & No errors detected \\
      \emph{pixel-queues-msq} & 01:00:00 & 3878 & 17847 & No errors detected \\
      \emph{pixel-queues-array} & 00:03:32 & 4146 & 8101 & Out of memory \\
      \emph{split} & 01:00:01 & 1113 & 3285 & No errors detected \\
      \emph{line-queues-msq} & 01:00:03 & 3423 & 9577 & No errors detected \\
      \emph{line-queues-array} & 01:00:03 & 3937 & 10665 & No errors detected \\
      \emph{split-counting} & 01:00:03 & 1070 & 3628 & No errors detected \\
      \emph{sequential} & 01:00:05 & 15 & 101 & No errors detected \\
      \emph{bogus-union} & 00:02:18 & 14 & 99 & \inlinecode{union} violates post-condition. \\
      \emph{bogus-loop} & 00:02:16 & 14 & 98 & Not monotonically decreasing. \\
      \hline
      \hline
    \end{tabular}
  } % makebox
  \caption[Verification results for the various area opening algorithms.]{Verification results for the various area opening algorithms. Time is formatted in \emph{hours:minutes:seconds}. The algorithms are denoted after table~\ref{tab:algorithms-acronyms}.}
  \label{tab:correctness-area-opening-verification-results}
\end{table}

All algorithms were model checked, providing the JVM with 4 gigabyte of memory
(except for \emph{bogus-union} and \emph{bogus-loop}, which were checked on a
different machine with only 2 gigabyte). The algorithms use the optimistic
fine-grained locking union-find variant, except for the bogus and the linear
implementation, which use the original algorithm. The parallel algorithms
utilize only two independent threads to reduce the number of possible
interleavings. Otherwise, the check would suffer from an immense state space
explosion. The time limit is set to one hour. If JPF has model-checked an
algorithm for an hour without any result, it terminates with the message ``No
errors detected''.

The collected results of the model-checking are shown in
table~\ref{tab:correctness-area-opening-verification-results}. The algorithms
are denoted with acronyms after table~\ref{tab:algorithms-acronyms}. All
parallel algorithms pass the model check for the correctness properties, as
presented in section~\ref{sec:correctness-area-opening-properties}, except for
the obviously wrong implementations. We can see that only a few states are
required to recognize a violation of the correctness properties.

Another important result is that JPF runs out of memory in case of
\emph{pixel-queues-array}. Still, JPF explores more states than for the
\citet{Michael1996Simple} queue variant, even though it proceeds not as deep
into the state space. It is not able to find any property violation.

Based on these results we can conclude that model checking is not necessarily
the best method to verify complex parallel algorithms. JPF has mostly been used
for concurrent applications, that have a somewhat simpler control flow and a
much smaller input space \cite{Visser2003Model, Khurshid2003Generalized}. Other
formal methods, for example the application of CSP implementations
\cite{Hoare1978Communicating, Roscoe1997Theory}, could provide a true proof of
correctness. These would, of course, require much more manual work. From this
point of view, model checking is the right choice, since it can be executed
automatically, because the correctness properties for union-find based area
opening are universal to this algorithm family. The results underline that it is
unfeasible to model check the algorithms without a time limit. Given the state
space of $O(nn!)$ and the fact that JPF can, according to the results, maximally
check roughly 25000 states per hour, depending on the algorithm, it would take
more than just days to verify the correctness of all programs: already for an
image of $3 \times 3$ pixels, we can expect an accumulated checking time of
about 43 days. While it would, in principle, still be possible to do this, it
is, due to a limit of resources accessible during this thesis, not possible to
model check the algorithms for several weeks.

%%% Local Variables:
%%% mode: latex
%%% TeX-master: "main"
%%% End:
