\chapter{Conclusion}
\label{chpt:conclusion}

In this thesis, I presented a variety of parallel algorithms to compute
morphological area opening based on the sequential, union-find based algorithm
by \citet{Meijster2002Comparison}. In particular, one of these parallel
algorithms -- namely parallel sub-image area opening with thread local pixel
sorting, which is based on a concurrent union-find data structure, that uses a
fine-grained locking scheme -- proved to outperform the sequential algorithm on
irregularly structured and flat images. Nevertheless, there exist at least two
pathological cases of input images, which let the performance of the new
parallel algorithm deteriorate. Future work should aim at finding a solution to
this. Additionally, I defined formal correctness conditions and showed that
model checking the new, parallel area opening algorithms through Java Pathfinder
\cite{Visser2003Model} does not reveal any errors. JPF identified bugs in
deliberately buggy implementations, which suggests that the here developed
algorithms are correct.

Furthermore, I provided a detailed discussion of the original sequential
algorithm, as well as its formalization and examples of its filtering effect on
images. Additionally, I identified parallel union-find algorithms that can be
mapped to area opening. The various union-find algorithms have been examined in
detail and their performance has been assessed independently on differently
structured input graphs. This showed that wait-free union-find and a mixed
wait-free and STM union-find algorithm outperform optimistic, fine-grained
locking union-find on graphs of high connectivity, since lock contention on a
single giant connected component is simply too high.

In contrast to this, we saw that these limitations do not necessarily apply to
parallel sub-image area opening, because here, each thread starts its work on
pixels on the image, which, on the array of image pixels, are located ``far
away'' from each other. Extensive lock contention is limited to a small portion
of work at the end of the filtering process. In the case of parallel area
opening, STM implementations of union-find exhibit systematically lower
performance. This might be due to the chosen STM library and further research on
this is needed.

This thesis opens up for many future research directions. While some of these
suggestions focus on validation and extension of the results presented in this
thesis, others focus on technical improvements. I believe that both are highly
relevant for parallel area opening to become a fast and reliable tool for use in
image analysis.

%%% Local Variables:
%%% mode: latex
%%% TeX-master: "main"
%%% End:
